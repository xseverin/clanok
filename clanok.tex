% Metódy inžinierskej práce

\documentclass[10pt,twoside,slovak,a4paper]{article}

%\usepackage[slovak]{babel}
%\usepackage[T1]{fontenc}
\usepackage[IL2]{fontenc} % lepšia sadzba písmena Ľ než v T1
\usepackage[utf8]{inputenc}
\usepackage{graphicx}
\usepackage{url} % príkaz \url na formátovanie URL
\usepackage{hyperref} % odkazy v texte budú aktívne (pri niektorých triedach dokumentov spôsobuje posun textu)
\usepackage{csquotes}
\usepackage{cite}
\usepackage{multirow}
%\usepackage{times}

\pagestyle{headings}

\title{Strategická hra Go\thanks{Semestrálny projekt v predmete Metódy inžinierskej práce, ak. rok 2022/23, vedenie: Ing. Vladimír Mlynarovič, PhD.}}

\author{Yevhenii Severin\\[2pt]
	{\small Slovenská technická univerzita v Bratislave}\\
	{\small Fakulta informatiky a informačných technológií}\\
	{\small \texttt{xseverin@stuba.sk}}
	}

\date{\small \today}



\begin{document}

\maketitle

\begin{abstract}
Go je dosková hra pochádzajúca zo starovekej Číny, ktorá sa s rozvojom techniky preniesla aj do počítačov. Existuje obrovské množstvo mobilných aplikácií, počítačových programov a stránok venovaných tejto hre, na ktorých sa môžete hrať s počítačom alebo s ľuďmi, prípadne sledovať, ako hrajú ostatní a získavať skúsenosti. Účelom článku je vysvetliť, čo táto hra ľuďom prináša, ako ju hrať, aké zručnosti rozvíja a stručne zhrnúť jej vývoj počas tisícročí. Na záver uvediem, ktoré stránky, programy či aplikácie sú pre túto hru najvhodnejšie.
\end{abstract}



\section{Úvod}
Počítačový Go program AlphaGo od DeepMind vyhral zápas proti jednému z najlepších svetových hráčov Lee Sedolovi z Kórey. Víťazstvo prišlo oveľa skôr, než ktokoľvek očakával, a prekvapilo mnohých v oblasti umelej inteligencii. Nolan Bushnell, zakladateľ Atari a sám Go Guru, bol tak ohromený počinom AlphaGo: \enquote{Go je najdôležitejšia hra v mojom živote. Je to jediná hra, ktorá skutočne vyvažuje ľavú a pravú stranu mozgu.  Skutočnosť, že teraz ustúpila počítačovej technológii, je nesmierne dôležitá}. Výhra počítača nad človekom tiež vyvolala obrovský záujem verejnosti o technológiu umelej inteligencii na celom svete, najmä v Číne, Kórei, USA a Spojenom kráľovstve. Pre mnohých ľudí má IT od tohto momentu nový význam. Ako povedal sám odporca tohto počítačového programu, víťaz 18-tich svetových titulov Go: \textquote{Myslel som, že AlphaGo je založený na výpočte pravdepodobnosti a že je to iba stroj. Ale keď som videl tento ťah, zmenil som názor. AlphaGo je určite kreatívny}. Neznamená to len informačné technológie alebo priemyselné technológie, teraz sú to inteligentné technológie a prichádza vek nových IT.\cite{webAlphaGo}\cite{7471613}
  
Prečo víťazstvo počítača nad človekom v tejto hre tak mnohých prekvapilo?

Myslím si, že dôvodom je to, že sme ľudia, myslíme si o sebe, že sme veľmi kreatívni. Totiž v strategickej hre Go sa tieto vlastnosti prejavujú. Prečo a ako táto hra rozvíja kreativitu a v akej forme je najlepšie ju hrať - to sú hlavné otázky tohto článku.

%Motivujte čitateľa a vysvetlite, o čom píšete. Úvod sa väčšinou nedelí na časti.
%Uveďte explicitne štruktúru článku. Tu je nejaký príklad.
%Základný problém, ktorý bol naznačený v úvode, je podrobnejšie vysvetlený v časti~\ref{nejaka}.
%Dôležité súvislosti sú uvedené v častiach~\ref{dolezita} a~\ref{dolezitejsia}.
%Záverečné poznámky prináša časť~\ref{zaver}.


\section{Všeobecné informácie o strategickej hre Go}
\subsection{História strategickej hry Go}
Hra je formou zábavy a vzrušenia, ktoré ľudia sledujú od úsvitu ľudskej civilizácie.\cite{9231758}

Strategická hra Go je považovaná za najstaršiu stolovú hru, ktorá sa nepretržite hrá až do súčasnosti a bola vytvorená v Číne pred viac ako 2500 rokmi. Historické anály Zuo Zhuan (4. storočie pred Kristom), ktoré sa týkajú historickej udalosti z roku 548 pred Kristom, sú najstaršou písomnou zmienkou o hre Go.\cite{9231758} 

Legendy vystopovali pôvod hry k mýtickému čínskemu cisárovi Yaoovi (2337 - 2258 p.n.l.), o ktorom sa hovorilo, že jeho poradca Shun navrhol hru Go pre jeho neposlušného syna Danzhu, aby naňho získal priaznivý vplyv.\cite{9231758}

\subsection{Pravidlá strategickej hry Go}\label{Pravidlá strategickej hry Go}
Niektoré stolové hry sú založené na čistej stratégii, ale mnohé obsahujú prvok náhody a niektoré dokonca čisto založené na náhode, bez prvku zručnosti. Strategická hra Go je jednou z takýchto stolových hier a je zvyčajne známa ako abstraktná strategická stolová hra pre dvoch hráčov, ktorej cieľom je obklopiť väčšie územie ako súper.\cite{9231758}

Go sa hrá s čiernymi a bielymi kameňmi na doske s 19 x 19 pretínajúcimi sa čiarami, hoci niekedy sa používajú dosky 13 x 13 a 9 x 9. Prvý ťah vykonáva hráč z čiernym kameňom a hráči sa striedajú buď položením jedného kameňa na jednu z prázdnych priesečníc, alebo vzdaním sa ťahu. Po umiestnení sa kameň nehýbe, avšak bloky kameňov môžu byť obkľúčené a zajaté protihráčom. Hra končí, keď obaja hráči vzdali ťahu.\cite{10.1145/792548.611939}

V turnajovej hre nedochádza k žiadnym remízam, pretože biely hráč získa 5,5 dodatočných bodov (nazývaných komi) ako kompenzáciu za to, že hrá ako druhý.\cite{10.1145/792548.611939}

\subsection{Analyzá strategickej hry Go}
Pravidlá strategickej hry Go(čast ~\ref{Pravidlá strategickej hry Go}) sú veľmi jednoduché, avšak zvládnuť hru samotnú je náročné. Počet teoreticky možných hier je rádovo $10^{700}$.\cite{10.1145/792548.611939}

Súčasťou toho, čo robí strategický hry Go tak zaujímavou a náročnou, je súhra medzi taktikou a stratégiou. Hráči bojujú v miestnych bitkách o územie v malých oblastiach hracej plochy a súčasné usporiadanie kameňov na jednej strane dosky ovplyvňuje pevnosť a užitočnosť kameňov na druhej strane dosky.\cite{7471613}

\section{ Najvhodnejšie stránky, programy a aplikácie pre strategickú hru Go }
\subsection{Stránky pre strategický hry Go}
Podľa môjho názoru sú pre potreby štúdia strategickej hry Go najvhodnejšie stránky amerických a britských Go asociácií a taktiež aj slovenská stránka asociácií Go je dobra. \cite{britGo}\cite{usGo}\cite{sGo}

Na týchto stránkach môžete nájsť obrovské množstvo dokumentácie o tejto hre a nápadov, ktoré Vám pomôžu zlepšiť vaše zručnosti.\\

\begin{tabular}{l|l}
\multicolumn{2}{c} {\textbf{Softvér na hranie Go na internete}}\\
Nazov & Popis\\
\hline
\multirow{3}{*}{Cgoban} & Klient pre KGS Go. \\&Obsahuje editor súborov SGF\\& na nahrávanie a prezeranie herných záznamov\\
\hline
Fox Go Server & Jeden z najpopulárnejších serverov\\
\hline
\multirow{2}{*}{Pandanet} & Klient IGS, ktorý umožňuje sledovať a hrať hry online \\&pre mobil alebo Windows/Mac/Linux\\
\hline
Tygem & Ďalší populárny online server\\
\hline
\multirow{2}{*}{Interactive Way To Go} & učí pravidlá Go a niečo \\&viacej o stratégii hry interaktívnym spôsobom
\end{tabular}

\cite{usGo}\cite{sGo}
\subsection{Programy a aplikácie pre strategický hry Go}
\begin{tabular}{l|l}
Nazov & Popis\\
\multicolumn{2}{c} {\textbf{K dispozícii na nákup}}\\
\hline
Crazy Stone & Go softwérové nástroje. Dostupé pre PC, iOS a Android.\\
\hline
\multirow{2}{*}{Many Faces of Go} & Ocenený program D. Fotlanda. \\&Igowin, 9x9 freeware demo, je k dispozícii na stiahnutie.\\
\hline
\multirow{2}{*}{SmartGo} & Niekoľko programov vrátane SmartGoKifu, \\&SmartGo Player, SmartGo pre Windows/Mac\\
\hline
\multicolumn{2}{c} {\textbf{Grafické používateľské rozhrania (bez nákladov)}} \\
\hline
\multirow{2}{*}{GoRilla} & produkt Windows, ktorý načítava, \\&upravuje a ukladá súbory SGF.\\
\hline
\multirow{3}{*}{KaTrain} & Grafické používateľské rozhranie spárované s KataGo,\\& ktoré je veľmi kreatívne \\&a inšpirujúce pri hraní a analýze hier s robotom.\\
\hline
\multirow{2}{*}{Lizzie} & Analytický program, ktorý je užitočným grafickým\\& rozhraním pre LeelaZero, KataGo alebo Pachi.\\
\hline
\multirow{2}{*}{PANDA-glGo} & 3D a 2D Goban, prehliadač a editor hier, \\& klient pre IGS-PandaNet a rozhranie pre GNU Go.\\
\hline
\multirow{2}{*}{q5Go} & Editor Windows SGF, klient IGS a rozhranie GTP, \\&ktoré podporuje LeelaZero a KataGo.\\
\hline
\multirow{2}{*}{Sabaki} & Veľmi pekné grafické používateľské rozhranie\\& pre Windows, Mac a Linux.\\
\hline
\multicolumn{2}{c} {\textbf{Roboty (bez nákladov)}}\\
\hline    
GnuGo & Open source go program.\\ 
\hline
\multirow{2}{*}{KataGo} & dokáže správne analyzovať hry pri znevýhodnení \\& a poskytnúť odhady skóre.\\
\hline
\multirow{3}{*}{Leela} & kombinuje pokroky v programovaní Go \\& a ďalší originálny výskum malého, \\&ľahko použiteľného grafického rozhrania.\\
\hline
\multirow{2}{*}{LeelaZero}& existuje veľa pekných editorov gtp, \\&ktoré umožňujú hrať a kontrolovať hry s LeelaZero.\\
\hline
Pachi& nemusí byť spustený na veľmi dobrom hardvéri.\\
\hline
\end{tabular}
\begin{tabular}{l|l}
\multicolumn{2}{c} {\textbf{Softvér, ktorý pomôže študovať (platené databázy)}}\\
\hline
BiGo & Databáza 100 000+ profesionálnych a 3 000 000 amatérskych  hier.\\
\hline
\multirow{2}{*}{GoGoD} & Databáza 98 000+ hier. Súčasťou je voliteľný program, \\&ktorý funguje iba so systémom Windows.\\
\hline
\multirow{2}{*}{go4go} & Databáza s viac ako 86 000 hrami\\& a týždennými aktualizáciami profesionálnych hier.\\
\hline

\end{tabular}
\cite{usGo}
%Z obr.~\ref{f:rozhod} je všetko jasné. 

%\begin{figure*}[tbh]
%\centering
%\includegraphics[scale=1.0]{diagram.pdf}
%Aj text môže byť prezentovaný ako obrázok. Stane sa z neho označný plávajúci objekt. Po vytvorení diagramu zrušte znak \texttt{\%} pred príkazom \verb|\includegraphics| označte tento riadok ako komentár (tiež pomocou znaku \texttt{\%}).
%\caption{Rozhodujúci argument.}
%\label{f:rozhod}
%\end{figure*}


%Základným problémom je teda\ldots{} Najprv sa pozrieme na nejaké vysvetlenie (časť~\ref{ina:nejake}), a potom na ešte nejaké (časť~\ref{ina:nejake}).\footnote{Niekedy môžete potrebovať aj poznámku pod čiarou.}

%Môže sa zdať, že problém vlastne nejestvuje\cite{Coplien:MPD}, ale bolo dokázané, že to tak nie je~\cite{Czarnecki:Staged, Czarnecki:Progress}. Napriek tomu, aj dnes na webe narazíme na všelijaké pochybné názory\cite{PLP-Framework}. Dôležité veci možno \emph{zdôrazniť kurzívou}.


% \subsection{Nejaké vysvetlenie} \label{ina:nejake}

%\paragraph{Veľmi dôležitá poznámka.}
%Niekedy je potrebné nadpisom označiť odsek. Text pokračuje hneď za nadpisom.

\section{Záver} \label{zaver} % prípadne iný variant názvu



%\acknowledgement{Ak niekomu chcete poďakovať\ldots}


% týmto sa generuje zoznam literatúry z obsahu súboru literatura.bib podľa toho, na čo sa v článku odkazujete
\bibliography{literatura}
\bibliographystyle{plain} % prípadne alpha, abbrv alebo hociktorý iný
\end{document}